\documentclass{article}
\usepackage[english]{babel}
\usepackage{amsmath,amssymb,graphicx,enumerate,latexsym}

%%%%%%%%%% Start TeXmacs macros
\newcommand{\dueto}[1]{\textup{\textbf{(#1) }}}
\newcommand{\longupdownarrow}{{\mbox{\rotatebox[origin=c]{-90}{$\longleftrightarrow$}}}}
\newcommand{\nobracket}{}
\newcommand{\tmaffiliation}[1]{\\ #1}
\newcommand{\tmemail}[1]{\\ \textit{Email:} \texttt{#1}}
\newcommand{\tmop}[1]{\ensuremath{\operatorname{#1}}}
\newenvironment{enumeratenumeric}{\begin{enumerate}[1.] }{\end{enumerate}}
\newenvironment{proof}{\noindent\textbf{Proof\ }}{\hspace*{\fill}$\Box$\medskip}
%%%%%%%%%% End TeXmacs macros

\begin{document}



\title{Topology}

\author{
  Liangchun Xu
  \tmaffiliation{Department of Mechanical Engineering, Tufts University\\
  574 Boston Avenue, Medford, 02155, US}
  \tmemail{liangchun.xu@tufts.edu}
}

\date{March 4, 2019}

\maketitle

{\tableofcontents}

\section{Cardinality and the Axiom of Choice}

{\definition*{{\dueto{number sets}}Throughout these notes we will use the
following notation:
\begin{enumeratenumeric}
  \item $\mathbb{N}= \tmop{the} \tmop{set} \tmop{of} \tmop{natural}
  \tmop{numbers}$ (i.e., the positive integers).
  
  \item $\mathbb{Z}= \tmop{the} \tmop{set} \tmop{of} \tmop{all}
  \tmop{integers}$.
  
  \item $\mathbb{Q}= \tmop{the} \tmop{set} \tmop{of} \tmop{rational}
  \tmop{numbers}$.
  
  \item $\mathbb{R}= \tmop{the} \tmop{set} \tmop{of} \tmop{real}
  \tmop{numbers}$.
\end{enumeratenumeric}}}

{\definition*{{\dueto{cardinality}}Two sets, $A \tmop{and} B$, have the same
cardinality iff there is a 1-1, onto function $f : X \rightarrow \{ 1, 2,
\ldots, n \}$ where $n$ is an element of $\mathbb{N}$. A set that's not finite
is infinite.}}

About the natural numbers $\mathbb{N}$: every non-empty set of natural numbers
has a least element.

{\theorem*{The even positive integers have the same cardinality as the natural
numbers.}}

\begin{proof}
  The even positive integers are
  \begin{eqnarray*}
    \mathbb{E}^+ & = & \{ \nobracket 2 n | n \in \mathbb{N} \}
  \end{eqnarray*}
  Construct function $f : \mathbb{N} \rightarrow \mathbb{E}^+$
  \[ f (n) = 2 n \]
  $f (n)$ is in $\mathbb{E}^+$ by definition. Claim $f$ is 1-1 and onto:
  
  1-1: Suppose $f (n) = f (n')$, then $2 n = 2 n'$ and $n = n'$;
  
  onto: Suppose $b \in \mathbb{E}^+$, then $b = 2 n$ for some $n \in
  \mathbb{N}$, $f (n) = 2 n = b$.
  
  As we have exhibited a $1 - 1$ onto function $f : \mathbb{N} \rightarrow
  \mathbb{E}^+$, $| \mathbb{N} | = | \mathbb{E}^+ |$.
\end{proof}

{\theorem*{$| \mathbb{N} | = | \mathbb{Z} |$.}}

\begin{proof}
  The correpondence is
  \[ \begin{array}{ccccccccccc}
       0 & - 1 & 1 & - 2 & 2 & - 3 & 3 & - 4 & 4 & - 5 & \ldots\\
       \longupdownarrow &  &  &  & \longupdownarrow &  &  &  &  &
       \longupdownarrow & \\
       1 & 2 & 3 & 4 & 5 & 6 & 7 & 8 & 9 & 10 & \ldots
     \end{array} \]
  
  
  Construct function $f : \mathbb{N} \rightarrow \mathbb{Z}$
  \[ f (n) = \left\{\begin{array}{l}
       \frac{n - 1}{2}, \tmop{if} n \in \mathbb{E}^+\\
       - \frac{n}{2}, \tmop{if} n \in \mathbb{O}^+
     \end{array}\right. \]
  $f (n)$ is in $\mathbb{Z}$ by definition. Claim $f$ is 1-1 and onto:
  
  1-1: Suppose $f (n) = f (n')$, then
  \[ \left\{\begin{array}{l}
       \frac{n - 1}{2} = \frac{n' - 1}{2}, \tmop{if} n \in \mathbb{E}^+\\
       - \frac{n}{2} = - \frac{n'}{2}, \tmop{if} n \in \mathbb{O}^+
     \end{array}\right. \Rightarrow n = n' \]
  Since $\frac{n - 1}{2} \geqslant 0 \tmop{and} - \frac{n}{2} < 0$, we have
  $\frac{n - 1}{2} \neq - \frac{n'}{2}$; only the above two cases are
  possible.
  
  onto: Suppose $b \in \mathbb{Z}$, then
  \[ \left\{\begin{array}{l}
       b = \frac{n - 1}{2}, \tmop{if} b \geqslant 0\\
       b = - \frac{n}{2}, \tmop{if} b < 0
     \end{array}\right. \]
  for some $n \in \mathbb{N}$, $f (n) = b$.
  
  As we have exhibited a $1 - 1$ onto function $f : \mathbb{N} \rightarrow
  \mathbb{Z}$, $| \mathbb{N} | = | \mathbb{Z} |$.
\end{proof}

{\theorem*{{\dueto{pigeon-hole principle}}Suppose that $n$ is a natural number
and you have $n$ pidgeon-holes. If you have $m$ podgeons where $m > n$, and
you put each pidgeon in one of your $n$ pidgeon-holes, there will be some hole
which contains more than 1 pidgeon.}}

\begin{proof}
  Let $q_i$ be the number of pidgeons in each hole, then
  \[ \sum_{i = 1}^n q_i = m \]
  Suppose no hole contains more than 1 pidgeon, which is
  \[ 0 \leqslant q_i \leqslant 1, \applicationspace{1 \tmop{em}} i = 1, 2,
     \ldots n \]
  Then we have
  \[ \sum_{i = 1}^n q_i \leqslant \sum_{i = 1}^n 1 = n \]
  Since $m > n$, then
  \[ \sum_{i = 1}^n q_i < m \]
  which leads to contradiction. Therefore the assumption is not correct. There
  must be some hole which contains more than 1 pidgeon.
\end{proof}

{\theorem*{{\dueto{induction}}For each natural number $n$, let $S (n)$ be a
statement that is either true or false. Then suppose that
\begin{enumeratenumeric}
  \item $S (1)$ is true.
  
  \item If $S (k)$ is true, then $S (k + 1)$ is true.
\end{enumeratenumeric}
Then $S (n)$ is true for all natural numbers $n$.}}

\begin{proof}
  Suppose there is an non-empty set
  \[ X = \{ n \in \mathbb{N}, S (n) \tmop{is} \tmop{false} \} \]
  There exists a smallest element $x \in X$ since $X \subseteq \mathbb{N}$.
  
  According to assumption 1 and the definiton of $X$, $x \neq 1 \tmop{and} x
  > 1$.
  
  According to assumption 2, if $S (x)$ is false, then $S (x - 1)$ is also
  false. Since $x - 1 < x$ and $x - 1 \in \mathbb{N}$, $x - 1$ becomes the
  smallest element in $X$, which is contradictory to the fact that $x$ is the
  smallest element in $X$. Therefore the assumption is wrong. $X$ is empty and
  $S (n)$ is true for all natural numbers $n$.
\end{proof}

{\theorem*{Every subset of $\mathbb{N}$ is either finite or has the same
cardinality as $\mathbb{N}$.}}

\begin{proof}
  Let $X \subseteq \mathbb{N}$, construct function $f : \mathbb{N} \rightarrow
  X$
  \begin{eqnarray*}
    f (1) & = & \min (X)\\
    f (2) & = & \min (X\backslash \{ f (1) \})\\
    & \vdots & \\
    f (n) & = & \min (X\backslash \{ f (1), f (2), \ldots, f (n - 1) \})
  \end{eqnarray*}
  if $X\backslash \{ f (1), f (2), \ldots, f (n - 1) \}$ is empty, then we
  have a 1-1 bijection $f : \{ 1, 2, \ldots, n \} \rightarrow X$, which means
  $X$ is finite. If $X$ is infinite, then claim $f : \mathbb{N} \rightarrow X$
  is $1 - 1$ and onto:
  
  1-1: Suppose $f (n) = f (n')$, then $n = n'$.
  
  onto: Suppose $b \in X$, then $b = \min (X\backslash \{ f (1), f (2),
  \ldots, f (n - 1) \})$ for some $n \in \mathbb{N}$, $f (n) = b$.
  
  As we have exhibited a $1 - 1$ onto function $f : \mathbb{N} \rightarrow X$,
  $| \mathbb{N} | = | X |$.
  
  Therefore every subset of $\mathbb{N}$ is either finite or has the same
  cardinality as $\mathbb{N}$.
\end{proof}

{\definition*{{\dueto{countable set}}A set that has the same cardinality as a
subset of $\mathbb{N}$ is countable.}}

{\theorem*{Every infinite set has a countably infinite subset.}}

{\theorem*{A set is infinite iff there is a 1-1 function from the set into a
proper subset of itself.}}

{\theorem*{$\mathbb{Q}$ is countable.}}

\begin{proof}
  Construct function $f : \mathbb{N} \rightarrow \mathbb{Q}$
  
  
  \[ \begin{array}{lllllllllllll}
       & 0 &  & - 1 &  & 1 &  & - 2 &  & 2 &  & - 3 & \ldots\\
       1 & \frac{0}{1} & \rightarrow & - \frac{1}{1} & \rightarrow &
       \frac{1}{1} &  & - \frac{2}{1} & \rightarrow & \frac{2}{1} &  & -
       \frac{3}{1} & \ldots\\
       &  &  &  & \swarrow &  & \nearrow &  & \swarrow &  & \nearrow &  & \\
       2 &  &  & - \frac{1}{2} &  & \frac{1}{2} &  & - \frac{2}{2} &  &
       \frac{2}{2} &  &  & \\
       &  &  & \downarrow & \nearrow &  & \swarrow &  & \nearrow &  &  &  &
       \\
       3 &  &  & - \frac{1}{3} &  & \frac{1}{3} &  & - \frac{2}{3} &  &  &  & 
       & \\
       &  &  &  & \swarrow &  & \nearrow &  &  &  &  &  & \\
       4 &  &  & - \frac{1}{4} &  & \frac{1}{4} &  &  &  &  &  &  & \\
       &  &  & \downarrow & \nearrow &  &  &  &  &  &  &  & \\
       5 &  &  & - \frac{1}{5} &  &  &  &  &  &  &  &  & \\
       \vdots &  &  &  &  &  &  &  &  &  &  &  & 
     \end{array} \]
  
  
  $f (n)$ is in $\mathbb{Z}$ by definition. Claim $f$ is 1-1 and onto:
  
  1-1: Suppose $f (n) = f (n')$, then
  
  
  
  onto: Suppose $b \in \mathbb{Z}$, then
  
  \
  
  for some $n \in \mathbb{N}$, $f (n) = b$.
  
  As we have exhibited a $1 - 1$ onto function $f : \mathbb{N} \rightarrow
  \mathbb{Q}$, $| \mathbb{N} | = | \mathbb{Q} |$.
\end{proof}

\end{document}
